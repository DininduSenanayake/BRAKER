\documentclass[a4paper,10pt]{report}
\usepackage[utf8]{inputenc}
\usepackage{hyperref}
\usepackage{rotating}
\usepackage{amssymb}


%opening
\title{BRAKER1: Unsupervised RNA-Seq-Based Genome Annotation with GeneMark-ET and AUGUSTUS - \textbf{Supplementary}}
\author{Katharina J. Hoff, Simone Lange, Alexandre Lomsadze,\\ Mark Borodovsky and Mario Stanke}


\begin{document}

\maketitle

\tableofcontents

\chapter{Supplementary Results}

Accuracy results of BRAKER1 in unmasked genomes are shown in Table \ref{unmasked}.


\begin{table}[htb!]
\caption{Accuracy results of BRAKER1 in unmasked genomes of four model organisms. \label{unmasked}}
\begin{tabular}{lp{.9cm}p{.9cm}p{.9cm}p{.9cm}}\hline
 & \multicolumn{1}{c}{\textit{A.~thaliana}} &  \multicolumn{1}{c}{\textit{C.~elegans}} &  \multicolumn{1}{c}{\textit{D.~melanogaster}} &  \multicolumn{1}{c}{\textit{S.~pombe}}\\
 \hline
Gene sensitivity        & 63.9 & 55.0 & 68.4 & 77.0\\
Gene specificity        & 51.6 & 55.2 & 60.6 & 80.1\\
Transcript sensitivity  & 54.6 & 43.2 & 50.6 & 77.0\\
Transcript specificity  & 50.4 & 53.3 & 59.3 & 76.1\\
Exon sensitivity        & 82.5 & 79.8 & 73.7 & 83.0\\
Exon specificity        & 78.8 & 85.4 & 67.1 & 82.9\\
\hline
\end{tabular}
\end{table}

Table \ref{compare} shows  gene prediction accuracy of GeneMark-ET (\textit{ab initio}, first step in BRAKER1) and of AUGUSTUS (trained on GeneMark-ET predictions, using RNA-Seq read alignment information as extrinsic evidence).

\begin{sidewaystable}
\caption{Accuracy results of GeneMark-ET (\textit{ab initio} predictions) and AUGUSTUS (using RNA-Seq) in BRAKER1 on softmasked genomes.
The AUGUSTUS results are the ones reported for BRAKER1 in Table 1 of the main article.\label{compare}}
\begin{tabular}{lp{1.5cm}p{1.2cm}p{1.5cm}p{1.2cm}p{1.5cm}p{1.2cm}p{1.5cm}p{1.2cm}}\hline
 & \multicolumn{2}{c}{\textit{A.~thaliana}} &  \multicolumn{2}{c}{\textit{C.~elegans}} &  \multicolumn{2}{c}{\textit{D.~melanogaster}} &  \multicolumn{2}{c}{\textit{S.~pombe}}\\
 & \tiny{GeneMark-ET} & \tiny{AUGUSTUS} & \tiny{GeneMark-ET} & \tiny{AUGUSTUS} &  \tiny{GeneMark-ET} & \tiny{AUGUSTUS} & \tiny{GeneMark-ET} &\tiny{AUGUSTUS}\\
 \hline
Gene sensitivity        & 51.4 & 63.9 & 42.8 & 55.2 & 54.6 & 67.0 & 81.8 & 76.7\\
Gene specificity        & 50.5 & 51.9 & 42.2 & 55.6 & 55.0 & 61.0 & 84.2 & 79.9\\
Transcript sensitivity  & 43.4 & 54.6 & 32.8 & 43.4 & 39.8 & 49.8 & 81.8 & 76.7\\
Transcript specificity  & 50.5 & 50.7 & 42.2 & 53.7 & 55.0 & 59.8 & 84.2 & 75.8\\
Exon sensitivity        & 78.6 & 82.6 & 79.8 & 80.2 & 66.6 & 73.1 & 88.0 & 82.7\\
Exon specificity        & 77.4 & 79.0 & 78.6 & 85.5 & 62.0 & 67.0 & 88.4 & 82.4\\
\hline
\end{tabular}
\end{sidewaystable}

\chapter{Supplementary Methods}

\section{RNA-Seq Alignment}

RNA-Seq libraries were aligned against the respective genomes with TopHat2 version 2.0.11 \cite{TopHat2} using Bowtie2 version 2.2.2\\ \cite{Bowtie2}.

In order to determine library specific characteristics such as insert size, TopHat2 was first run with standard parameters. Initial alignments were processed with Cufflinks \cite{Cufflinks} version 2.2.0. The Cufflinks run was interrupted after the output showed ``Fragment Length Distribution'' parameters. Subsequently, TopHat2 was run with the such determined mean insert size and standard deviation. Table \ref{Tab:01} shows Tophat2 parameters that were used for all sets of RNA-Seq libraries in this publication.

\begin{sidewaystable}
\begin{center}
\begin{tiny}
 \begin{tabular}{p{0.2\textwidth} c c c c c c c}
 \hline
Library & \texttt{--mate-inner-dist} & \texttt{--mate-std-dev} & \texttt{--min-intron-length} & \texttt{--min-coverage-intron} & \texttt{--min-segment-intron} & \texttt{--microexon-search} & \texttt{--max-intron-length} \\
\hline
SRR934391 & 20 & 46 & -- &  -- & -- & \checkmark & 100000\\
SRR065719 & 14 & 25 &20&20&20& \checkmark & 100000\\
SRR023505, SRR023546, SRR023608, SRR026433, SRR027108 & 38 & 26 & 30 & 30 & 30 & \checkmark & --\\
SRR097898, SRR097899, SRR097900, SRR097902, SRR097903, SRR097905, SRR097906, SRR097907, SRR097908, SRR097909, SRR097912, SRR097915, SRR097917, SRR097921, SRR097922, SRR097925, SRR402833  & -- & -- & 20 & 20 & 20 & \checkmark & 100000\\
\hline
 \end{tabular}
 \end{tiny}
 \end{center}
\caption{\label{Tab:01}TopHat2 parameters used for RNA-Seq alignment.}
\end{sidewaystable}

\section{Cufflinks Assembly}

MAKER2 \cite{MAKER2} and CodingQuarry \cite{CodingQuarry} require an RNA-Seq assembly. We used Cufflinks version 2.2.0 with standard parameters to assemble the libraries that had previously been aligned to the genome with TopHat2. 

\section{Repeat Masking}

Genomes were softmasked for repeats using RepeatModeler 1.0.8 \cite{RepeatModeler}.

\section{Running BRAKER1}

The command for running BRAKER1 version 1.5 with GeneMark-ET version 4.28 and AUGUSTUS version 3.1.0 on softmasked genomes was\\

\noindent \texttt{braker.pl --genome=genome.fa --species=speciesname --bam=accepted\_hits.bam --softmasked}\\

\noindent where \texttt{accepted\_hits.bam} is the TopHat2 alignment file and \texttt{speciesname} is a name for storing output parameters.

\noindent For unmasked genomes, the command flag \texttt{--softmasked} was removed.

\noindent For \textit{S.~pombe}, BRAKER1 was run with the flag \texttt{--fungus} which enable usage of the fungi-specific branchpoint model of GeneMark-ET.

\section{Running MAKER2}

For running MAKER2, we followed the tutorial at \url{http://weatherby.genetics.utah.edu/MAKER/wiki/index.php/MAKER_Tutorial_for_GMOD_Online_Training_2014}, mostly.

MAKER2 was run with the gene finders SNAP \cite{SNAP}, AUGUSTUS \cite{AUGUSTUS}, and GeneMark-ES \cite{GeneMark-ES}. MAKER2 was developed to run with protein database to support gene discovery. We did not enable database support because BRAKER1 has no such feature. Thus, MAKER2 - in the way that we used it - is a method that calls and masks repeats, generates training gene structures on the basis of RNA-Seq data, and predicts genes with extrinsic evidence from RNA-Seq data. Running MAKER2 on a novel species consists of three steps: 

\begin{enumerate}
 \item Generating training gene structures for SNAP and AUGUSTUS with MAKER2,
 \item training SNAP, AUGUSTUS and GeneMark-ES (outside of MAKER2), and
 \item predicing genes with MAKER2 using GeneMark-ES, SNAP and AUGUSTUS.
\end{enumerate}

\noindent Training gene structures can be improved an iterative fashion. We computed two iterations.

Training of the gene prediction tool GeneMark-ES depends on the genome sequence, only, and is therefore independent of MAKER2. GeneMark-ES was trained using the command\\

\noindent \texttt{gm\_es.pl genome.fa}

\noindent For \textit{S.~pombe}, GeneMark-ES was trained using the flag \texttt{--fungus} to enable the fungi-specific branch point model in GeneMark-ES.

\subsection{Generating training gene structures for SNAP and AUGUSTUS (first iteration)} \label{training_genes_it1}

MAKER2 configuration files were generated\\

\noindent \texttt{maker -CTL}\\

\noindent The file \texttt{maker\_opts.ctl} was edited to contain (besides default parameters):

\begin{verbatim}
genome=genome.fasta
est=transcripts.fa # Cufflinks assembly
est2genome=1
\end{verbatim}

\noindent No HMMs for gene finders were configured. MAKER2 was run with the following command:\\

\noindent \texttt{maker}\\

\noindent Chromosome-specific \texttt{gff} files were converted to \texttt{ann/zff} and \texttt{dna} files (native format for training SNAP):

\begin{verbatim}
# maker2zff belongs to MAKER2
maker2zff Chr.gff
mv genome.ann genome.Chr.ann
mv genome.dna genome.Chr.dna
cat genome.Chr.ann ... > genome.ann
cat genome.Chr.dna ... > genome.dna
\end{verbatim}

\noindent To obtain a training gene file for AUGUSTUS, \texttt{ann/zff} files were first converted to \texttt{gff3}, and from there to \texttt{gtf}:

\begin{verbatim}
# zff2gff3.pl belongs to SNAP
zff2gff3.pl genome.Chr.ann > Chr.gff3
cat Chr.gff3 | perl -ne '
  if(not(m/^\#/)){
    chomp; @t = split(/\t/); 
    @t2 = split(/=/, $t[7]);
    print "$t[0]\t$t[1]\t$t[2]\t$t[3]\t$t[4]\t$t[5]\t$t[6]\t";
    print "\tgene_id \"$t2[1]Chr\"; transcript_id"; 
    print "  \"$t2[1]Chr\"\n";
  }'> Chr.gtf
# the last two lines makes gene/transcript IDs unique across 
# different chromosomes

# join gtf files from different chromosomes:
cat Chr.gtf ... > all.gtf
\end{verbatim}

\noindent AUGUSTUS training genes are excised with a flanking noncoding region. In BRAKER1, the average gene length divided by two is used as a flanking region length. The average gene length and resulting flanking region length were computed:

\begin{verbatim}
cat all.gtf | perl -ne '
  @t = split(/\t/);
  $seen{$t[8]} += ($t[4] - $t[3] + 1); 
  if(eof()){
    $sum = 0; $c = 0; 
    foreach my $key ( keys %seen ){
    $c=$c+1; $sum += $seen{$key};} 
    print $sum."/".$c."=".($sum/$c); 
    print "\n";
  }'
# the resulting number was divided by two
\end{verbatim}

\noindent This resulted in the following flanking region lengths that were used for excising training genes from the genome for training AUGUSTUS for MAKER2 (first iteration):\\

\begin{center}
\begin{tabular}{l r}
\hline
Species & Flanking region length (nt)\\
\hline
\textit{Arabidopsis thaliana} & 644\\
\textit{Caenorhabditis elegans} & 616\\
\textit{Drosophila melanogaster} & 972\\
\textit{Schizosaccharomyces pombe} & 1050\\
\hline
\end{tabular}
\end{center}

\noindent The \texttt{gtf} file was converted to a \texttt{gb} file with above flanking region length:\\

\begin{verbatim}
# gff2gbSmallDNA.pl belongs to AUGUSTUS
gff2gbSmallDNA.pl all.gtf genome.fasta $flanking_region_length first.gb
\end{verbatim}

\noindent Above described procedure lead to the generation of the following numbers of training genes:

\begin{center}
\begin{tabular}{l r}
\hline
Species & Number of training genes\\
\hline
\textit{Arabidopsis thaliana} & 13547\\
\textit{Caenorhabditis elegans} & 7660\\
\textit{Drosophila melanogaster} & 7049\\
\textit{Schizosaccharomyces pombe} & 307\\
\hline
\end{tabular}
\end{center}

\subsection{Training SNAP and AUGUSTUS (first iteration)}

\subsubsection{Training SNAP} \label{train_snap_it1}

\begin{verbatim}
# fathom, forge and hmm-assembler.pl are part of SNAP
fathom -categorize 1000 genome.ann genome.dna
fathom -export 1000 -plus uni.ann uni.dna
forge export.ann export.dna
hmm-assembler.pl ${species} . > ${species}.hmm
\end{verbatim}

\subsubsection{Training AUGUSTUS} \label{train_augustus_it1}

A new species for AUGUSTUS parameters was created:

\begin{verbatim}
# new_species.pl is part of AUGUSTUS
new_species.pl --species=${species}
\end{verbatim}


\noindent The original training gene structure contained errors, such as occasionally missing start- or stop-codons. Such error containing genes were filtered out:\\

\begin{verbatim}
# etraining and filgerGenesOut_mRNAname.pl are part of AUGUSTUS
etraining --species=maker2_spomb1 first.gb 1> etrain-test.out 
   2> etrain-test.err
   
fgrep "gene" etrain-test.err | cut -f 2 -d " " > bad.etraining-test.lst

filterGenesOut_mRNAname.pl bad.etraining-test.lst first.gb > second.gb

etraining --species=maker2_spomb1 second.gb
\end{verbatim}

\noindent The training gene set was split into two sets, the second set was subsequently further split in another two sets, resulting in three different files:

\begin{enumerate}
 \item A small ``test set'' of 200 (in case of \textit{S.~pombe} first iteration: 23) genes for measuring accuracy after \texttt{etraining} and \texttt{optimize\_augustus.pl}, 
 \item a large gene set for \texttt{etraining}, that was further split into:
 \begin{enumerate}
 \item a large gene set for for the option \texttt{--onlytrain} of \\\texttt{optimize\_augustus.pl},
 \item a small gene set for \texttt{optimize\_augustus.pl}, the size was 1000 for all species except for \texttt{S.~pombe}, where genes were not further split due to their small number.
 \end{enumerate}
\end{enumerate}

\begin{verbatim}
# randomSplit.pl is part of AUGUSTUS
randomSplit.pl second.gb 200
randomSplit.pl second.gb.train 1000
# this results in the following files:
# 1) second.gb.test -> measuring accuracy
# 2) second.gb.train -> etraining
# 2a) second.gb.train.train -> --onlytrain in optimize_augustus.pl
# 2b) second.gb.train.test -> optimize_augustus.pl
\end{verbatim}

\noindent Major AUGUSTUS parameters were ajusted with \texttt{etraining}:\\

\noindent \texttt{etraining --species=\$\{species\} second.gb.train}\\

%\noindent Accuracy after \texttt{etraining} was tested with:\\

%\noindent \texttt{augustus --species=\$\{species\} second.gb.test}\\

%\noindent Accuracy results after different training steps are summarized in table \ref{TrainAcc}.
\noindent Other parameters were optimized with \texttt{optimize\_augustus.pl:}\\

\noindent \texttt{optimize\_augustus.pl --species=\$\{species\} --onlytrain=second.gb.train.train second.gb.train.test}\\

%\noindent Accuracy after \texttt{optimize\_augustus.pl} was tested with:\\

%\noindent \texttt{augustus --species=\$\{species\} second.gb.test}\\

\subsection{Generating training gene structures (second iteration)}

\subsubsection{Generating training gene structures for SNAP}

MAKER2 parameters were generated:\\

\noindent \texttt{maker -CTL}\\

\noindent The file \texttt{maker\_opts.ctl} was edited to contain (besides default parameters):

\begin{verbatim}
genome=genome.fasta
est=transcripts.fa # Cufflinks assembly
snaphmm=${species}.hmm
\end{verbatim}

\noindent MAKER2 was run with the following command:\\

\noindent \texttt{maker}\\

\noindent Training gene extraction for SNAP was performed as described for iteration 1 in section \ref{training_genes_it1}.

\subsubsection{Generating training gene structures for AUGUSTUS}

MAKER2 parameters were generated:\\

\noindent \texttt{maker -CTL}\\

\noindent The file \texttt{maker\_opts.ctl} was edited to contain (besides default parameters):

\begin{verbatim}
genome=genome.fasta
est=transcripts.fa # Cufflinks assembly
augustus_species=${species}
\end{verbatim}

\noindent MAKER2 was run with the following command:\\

\noindent \texttt{maker}\\

\noindent Training gene extraction for AUGUSTUS was performed as described for iteration 1 in section \ref{training_genes_it1}, leading to the following numbers of training genes:

\begin{center}
\begin{tabular}{l r}
\hline
Species & Number of training genes\\
\hline
\textit{Arabidopsis thaliana} & 11887\\
\textit{Caenorhabditis elegans} & 5711\\
\textit{Drosophila melanogaster} & 7568\\
\textit{Schizosaccharomyces pombe} & 1013\\
\hline
\end{tabular}
\end{center}

\subsection{Training SNAP and AUGUSTUS (second iteration)}

\subsubsection{Training SNAP}

Training SNAP was performed as described in section \ref{train_snap_it1}.

\subsubsection{Training AUGUSTUS}

Training AUGUSTUS was performed as described in section \ref{train_augustus_it1}, except that no new species was created (parameters of iteration 1 were refined).

\subsection{Predicting genes with MAKER2}

\subsubsection{Preparing \texttt{rnaseq.gff3}}

Junctions generated by TopHat2 and Cufflinks transcripts were converted to \texttt{gff3} format:

\begin{verbatim}
# tophat2gff3 and cufflinks2gff3 are part of MAKER2
tophat2gff3 junctions.bed > tophat.gff3
cufflinks2gff3 transcripts.gtf > cufflinks.gff3
cat tophat.gff3 cufflinks.gff3 > rnaseq.gff3
\end{verbatim}


\subsubsection{Running MAKER2}

MAKER2 parameters were generated:\\

\noindent \texttt{maker -CTL}\\

\noindent The file \texttt{maker\_opts.ctl} was edited to contain (besides default parameters):

\begin{verbatim}
genome=genome.fasta
est=transcripts.fa # Cufflinks assembly
est_gff=rnaseq.gff3 # TopHat2 and Cufflinks
augustus_species=${species}
snaphmm=${species}.hmm #SNAP HMM file
gmhmm=${species}/gmhmm.mod #GeneMark HMM file
augustus_species=${species} # AUGUSTUS model
keep_preds=1
\end{verbatim}

\noindent MAKER2 was run with the following command:\\

\noindent \texttt{maker}\\

\section{Running CodingQuarry}

As a tool specific for fungi, we ran CodingQuarry on \textit{S.~pombe}, only. First, Cufflinks assemblies were converted from \texttt{gtf} to \texttt{gff3}, then CodingQuarry was executed:

\begin{verbatim}
CufflinksGTF_to_CodingQuarryGFF3.py transcripts.gtf > transcripts.gff3

CodingQuarry -f genome.fasta -t transcripts.gff3 -d -p 8
\end{verbatim}


\section{Measuring Accuracy}

Accuracy was measured using the Eval package \cite{Eval}.



\begin{thebibliography}{}
\bibitem[Kim {\it et~al}., 2013]{TopHat2} Kim, D. and Pertea, G. and Trapnell, C. and Pimentel, H. and Kelley, R. and Salzberg, S.L. (2013) TopHat2: accurate alignment of transcriptomes in the presence of insertions, deletions and gene fusions, {\it Genome Biology} \textbf{14}:R36.

\bibitem[Langmead and Salzberg, 2012]{Bowtie2} Langmead, B. and Salzberg, S.L. (2012) Fast gapped-read alignment with Bowtie 2, \textit{Nature Methods} \textbf{9}: 357-359.

\bibitem[Mortazavi {\it et~al}., 2008]{Cufflinks} Mortazavi, A. and Williams, B.A. and McCue, K. and Schaeffer, L. and Wold, B. (2008) Mapping and quantifying mammalian transcriptomes by RNA-Seq, \textit{Nature Methods} \textbf{5}: 621-628.

\bibitem[Holt and Yandell, 2011]{MAKER2} Holt, C. and Yandell, M. (2011) MAKER2: an annotation pipeline and genome-database management tool for second-generation genome projects, \textit{BMC Bioinformatics}, \textbf{12}:491.

\bibitem[Testa \textit{et~al}., 2015]{CodingQuarry} Testa, A.C. and Hane, J.K. and Ellwood, S.R. and Oliver R.P. (2015) CodingQuarry: highly accurate hidden Markov model gene prediction in fungal genomes using RNA-seq transcripts, \textit{BMC Genomics} \textbf{16}:170.

\bibitem[Korf, 2004]{SNAP} Korf, I. (2004) Gene finding in novel genomes, \textit{BMC Bioinformatics} \textbf{5}:59 S1-S9.

\bibitem[Stanke \textit{et~al}., 2008]{AUGUSTUS}
Stanke, M. and Diekhans, M. and Baertsch, R. and Haussler, D. (2008) Using native and syntenically mapped cDNA alignments to improve de novo gene finding, \textit{Bioinformatics}, \textbf{24}(5), 637.

\bibitem[Lomsadze \textit{et~al}., 2005]{GeneMark-ES} Lomsadeze, A. and Ter-Hovhannisyan, V. and Chernoff, Y. and Borodovsky, M. (2005) Gene identification in novel eukaryotic genomes by self-training algorithm, \textit{Nucleic Acids Research} \textbf{33}:20 6494-6506.

\bibitem[Keibler and Brent, (2003)]{Eval} Keibler, E. and Brent, M.R. (2003) Eval: a software package for analysis of genome annotations, \textit{BMC Bioinformatics} \textbf{4}:50.

\bibitem[Smit and Hubley, 2015]{RepeatModeler} Smit, A.F.A. and Hubley, R. (2008-2015) RepeatModeler Open-1.0 \textit{\url{http://www.repeatmasker.org}}.

\end{thebibliography}

\end{document}
